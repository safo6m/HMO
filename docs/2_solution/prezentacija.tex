\documentclass[utf8]{beamer}
\usetheme{Antibes}
\usepackage[T1]{fontenc}
\usepackage{tikz}
\usepackage{caption}
\usepackage{url}
\usepackage{listings}
\usepackage{color}

\lstset{
language=python,                % choose the language of the code
basicstyle=\footnotesize,       % the size of the fonts that are used for the code
numbers=left,                   % where to put the line-numbers
numberstyle=\footnotesize,      % the size of the fonts that are used for the line-numbers
stepnumber=1,                   % the step between two line-numbers. If it is 1 each line will be numbered
numbersep=6pt,                  % how far the line-numbers are from the code
backgroundcolor=\color{white},  % choose the background color. You must add \usepackage{color}
showspaces=false,               % show spaces adding particular underscores
showstringspaces=false,         % underline spaces within strings
showtabs=false,                 % show tabs within strings adding particular underscores
frame=L,           % adds a frame around the code
tabsize=2,          % sets default tabsize to 2 spaces
captionpos=b,           % sets the caption-position to bottom
breaklines=true,        % sets automatic line breaking
breakatwhitespace=false,    % sets if automatic breaks should only happen at whitespace
escapeinside={\%*}{*)}          % if you want to add a comment within your code
}

% Postavljanje fonta
\if@fonttimes\RequirePackage{times} \fi
\if@fontlmodern\RequirePackage{lmodern} \fi

\usecolortheme{beaver}

\newcommand{\engl}[1]{(engl.~\emph{#1})}

\title[Projekt]{Kapacitativni problem usmjeravanja vozila iz višebrojnih skladišta}
\author{Krešimir Baksa, Mihael Šafarić i Matija Šantl}

\institute{Heurističke metode optimizacije\\*Fakultet elektrotehnike i računarstva}
\date{Zagreb, siječanj 2015.}
\begin{document}

\begin{frame}
\titlepage
\end{frame}

% Naslovni slide s imenima članova tima
% Uvod
% Opis algoritma
% Pseudokod
% Vaše najbolje dobiveno rješenje instance problema

\section{Uvod}
\begin{frame}{Uvod}

Zadano:
\begin{itemize}
  \item usmjereni težinski graf
  \item skladišta, njihov položaj, kapacitet i pripadajući troškovi otvaranja skladišta 
  \item korisnici, njihov položaj i potražnja
  \item početni trošak i kapacitet vozila
\end{itemize}

Varijable:
\begin{itemize}
  \item broj skladišta 
  \item broj vozila
  \item obilasci vozila
\end{itemize}

Optimizacijski kriteriji:
\begin{itemize}
  \item minimizirati ukupni trošak otvaranja skladišta i ruta te ukupni trošak odabranih ruta
\end{itemize}

\end{frame}

\subsection{Pohlepni algoritam}
\begin{frame}
\frametitle{Pohlepni algoritam}

Algoritam nakon učitavanja podataka iterativno kombinira manji broj skladišta te pokušava pohlepnim pretraživanjem Hamiltonovih ciklusa izgraditi konačno rješenje. U svakoj iteraciji se gradi jedan Hamiltonov ciklus za svako skladište te se u konačno rješenje dodaje samo najkraći ciklus. Algoritam se zaustavlja ukoliko je problem rješen ili nema više dostupnih vozila.

\end{frame}

\begin{frame}
\frametitle{Izgradnja Hamiltonovog ciklusa}
Hamiltonov ciklus se izgrađuje dodavanjem najbližeg čvora u cuklus sve dok ukupan trošak čvorova ne premašuje kapacitet vozila. Ukoliko je na kraju ostalo dovoljno kapaciteta u vozilu pokuša se dodati i neki udaljeniji čvor koji ima manji trošak. Relativna udaljenost drugih čvorova određena je udaljenošću zadnja dva čvora u ciklusu.
\end{frame}

\begin{frame}[fragile]{Izgradnja Hamiltonovog ciklusa}
\begin{lstlisting}
dodaj skladiste u ciklus
prethodni cvor = skladiste
dok vozilo ima kapacitet i postoji korisnik:
  trenutni cvor = najblizi(prethodni cvor)  
  ako trenutni cvor ima preveliki trosak i postoji drugi cvor:
    trenutni cvor = pronadi drugi(prethodni cvor)
  inace:
    gotovo
  dodaj trenutni cvor u ciklus
  oduzmi kapacitet
  prethodni cvor = trenutni cvor
  
  
\end{lstlisting}
\end{frame}

\begin{frame}[fragile]{Izgradnja rješenja}
\begin{lstlisting}
dok ima vozila ili korisnika:
  za svako dostupno skladiste:
    ako ima vozila u skladistu:
      izgradi ciklus
    ako je ciklus najkraci:
      zapamti najkraci ciklus

  spremi najkraci ciklus u rjesenje
  osvjezi neposluzene korisnike
  oduzmi vozilo iz skladista

ako ima neposluzenih korisnika:
  rjesenje nije pronadeno
inace
  vrati rjesenje

\end{lstlisting}
\end{frame}

\section{Ostvareno rješenje}
\begin{frame}{Ostvareno rješenje}

Najbolje ostvareno rješenje iznosi 285727.

\end{frame}

\end{document}