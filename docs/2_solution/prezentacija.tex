\documentclass[utf8]{beamer}
\usetheme{Antibes}
\usepackage[T1]{fontenc}
\usepackage{tikz}
\usepackage{caption}
\usepackage{url}
\usepackage{listings}
\usepackage{color}

\lstset{
language=python,                % choose the language of the code
basicstyle=\footnotesize,       % the size of the fonts that are used for the code
numbers=left,                   % where to put the line-numbers
numberstyle=\footnotesize,      % the size of the fonts that are used for the line-numbers
stepnumber=1,                   % the step between two line-numbers. If it is 1 each line will be numbered
numbersep=6pt,                  % how far the line-numbers are from the code
backgroundcolor=\color{white},  % choose the background color. You must add \usepackage{color}
showspaces=false,               % show spaces adding particular underscores
showstringspaces=false,         % underline spaces within strings
showtabs=false,                 % show tabs within strings adding particular underscores
frame=L,           % adds a frame around the code
tabsize=2,          % sets default tabsize to 2 spaces
captionpos=b,           % sets the caption-position to bottom
breaklines=true,        % sets automatic line breaking
breakatwhitespace=false,    % sets if automatic breaks should only happen at whitespace
escapeinside={\%*}{*)}          % if you want to add a comment within your code
}

% Postavljanje fonta
\if@fonttimes\RequirePackage{times} \fi
\if@fontlmodern\RequirePackage{lmodern} \fi

\usecolortheme{beaver}

\newcommand{\engl}[1]{(engl.~\emph{#1})}

\title[Projekt]{Kapacitativni problem usmjeravanja vozila iz višebrojnih skladišta}
\author{Krešimir Baksa, Mihael Šafarić i Matija Šantl}

\institute{Heurističke metode optimizacije\\*Fakultet elektrotehnike i računarstva}
\date{Zagreb, siječanj 2015.}
\begin{document}

\begin{frame}
\titlepage
\end{frame}

% Naslovni slide s imenima članova tima
% Uvod
% Opis algoritma
% Pseudokod
% Vaše najbolje dobiveno rješenje instance problema

\section{Uvod}
\begin{frame}{Uvod}

Zadano:
\begin{itemize}
	\item usmjereni težinski graf
	\item skladišta, njihov položaj, kapacitet i pripadajući troškovi otvaranja skladišta 
	\item korisnici, njihov položaj i potražnja
	\item početni trošak i kapacitet vozila
\end{itemize}

Varijable:
\begin{itemize}
	\item broj skladišta 
	\item broj vozila
	\item obilasci vozila
\end{itemize}

Optimizacijski kriteriji:
\begin{itemize}
	\item minimizirati ukupni trošak otvaranja skladišta i ruta te ukupni trošak odabranih ruta
\end{itemize}

\end{frame}

\section{Opis algoritma}
\begin{frame}
\frametitle{Pohlepni algoritam}

Pohlepni algoritam nakon učitavanja podataka razvrstava korisnike po skladištima na način da u trenutnom koraku minimizira trošak trenutnog korisnika do njemu najbližeg iz skladište u koje ga pokušava staviti uz dodatak troška povratka od tog korisnika do skladišta. Ako skladište u tom trenutku nema ni jednog korisnika, cijeni se dodaje trošak otvaranja skladišta.

\vspace{5mm}

To radi tako dugo dok ne razvrsta sve korisnike.

\end{frame}

\begin{frame}
\frametitle{Mutacija}

Nakon dobivanja početnog rješenja, napravi se $POPULACIJA$ broj kopija tog rješenja, te se nad svakim od njih radi se mutacija.

Za slučajno odabrana skladišta $X$ i $Y$, slučajnim odabirom uzmi korisnika $Z$ iz skladišta $X$ te ga probaj staviti u skladište $Y$ ako time ne narušavamo kapacitet skladišta $Y$.

\vspace{5mm}

Ukoliko je mutirano rješenje bolje od trenutnog, ono se zamjenjuje u populaciji.

\vspace{5mm}

Taj se postupak ponavlja $ITERACIJA$ broj puta.

\end{frame}

\begin{frame}
\frametitle{Lokalna pretraga}

Nakon što smo dobili $POPULACIJA$ broj mutiranih rješenja, nad svakim od njih se pokreće postupak lokalne pretrage. 

\vspace{5mm}

Postupak lokalne pretrage, odabire par korisnika $X$ i $Y$, te zamjenjuje njihova skladišta. Ako je rješenje ne narušava kapacitet ni jednog skladišta, spremamo ga u red te nastavljamo lokalnu pretragu sa sljedećim rješenjem iz reda.

\vspace{5mm}

Postupak radimo tako dugo dok ima rješenja u redu ili napravimo $limit$ broj iteracija.

\end{frame}

\section{Pseudokod}
\begin{frame}[fragile]{Početno rješenje}
\begin{lstlisting}
dok ima nereazvrstanih korisnika X:
  za svako skladiste Y:
    za svakog korisnika Z u skladistu Y:	
		
      cijena = udaljenost(X, Z) + udaljenost(X, Y)
			
      ako je skladiste Y prazno:
        cijena += cijena otvaranja skladista Y
		
      ako je  cijena najmanja do sad:
        W = Y
	
      stavi korisnika X u skladiste W
\end{lstlisting}
\end{frame}

\begin{frame}[fragile]{Mutacija}
\begin{lstlisting}
ponovi ITERACIJA broj puta:
  za svako rjesenje R iz populacije:
    slucajno odaberi skladiste X
    slucajno odaberi skladiste Y
    slucajno odaberi korisnika A iz skladista X
		
    ako prebacivanje A iz X u Y narusava ogranicenja:
      ponovi odabir skladista i korisinka
    inace:
      prebaci korisnika A iz X u Y
      spremi izmjenjeno rjesenje
\end{lstlisting}
\end{frame}

\begin{frame}[fragile]{Lokalna pretraga}
\begin{lstlisting}
dodaj rjesenje X u red Q

ponovi LIMIT broj puta ili dok ne konvergira:
  rjesenje R = prvo rjesenje iz reda Q
  za svaki par korisnika (A, B):
    ako zamjenom skladista korisnika A i B u R ne narusavao ogranicenja:
      R` = zamjeni skladista korisnika A i B
      
      ako je R` bolje od X:
        X = R`
      
      ako je trenutno R` bolje R:
        dodaj R` u Q
\end{lstlisting}
\end{frame}

\begin{frame}[fragile]{Fitness}
\begin{lstlisting}
fitness = 0
za svako skladiste X:
  ako ima korisnika u X:
    fitness += cijena otvaranja skladista X
    
    pohlepnim algoritmom napravi potrebne rute R[] za sve korisnike iz X
    za svaku rutu R`:
      grubom silom izracunaj najkraci Hamiltonov ciklus, C
      cijena += obilazak ciklusa C
      cijena += cijena automobila * broj ruta
      
\end{lstlisting}
\end{frame}

\begin{frame}[fragile]{Rješenje}
\begin{lstlisting}
A = pocetno rjesenje
P = niz od POPULACIJA kopija od A

ponovi ITERACIJA broj puta:
  za svako rjesenje R iz P:
    mutiraj(R)

za svako rjesenjeR  iz P:
  lokalna_pretraga(R)
	
izdvoji najbolje rjesenje iz P
\end{lstlisting}
\end{frame}

\section{Ostvareno rješenje}
\begin{frame}{Ostvareno rješenje}

Najbolje ostvareno rješenje iznosi 315983.

\end{frame}

\end{document}